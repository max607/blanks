%%%%% preamble
\documentclass[12pt]{article}

% meta data
\title{Seminar Paper}
\author{Maximilian Schneider}
\date{\today}

% text encoding
\usepackage[utf8]{inputenc}

% better interaction with pdf (strg + f)
\usepackage[T1]{fontenc}

% comment, quote environment (load csquotes before babel)
\usepackage{comment, csquotes}

% language, line breaks
\usepackage[english]{babel}

% page measurements
\usepackage[a4paper, width=150mm, top=25mm, bottom=25mm, bindingoffset = 6mm]{geometry}

% more layout
\usepackage{fancyhdr}
\pagestyle{fancy}  % activate fancyhdr
\renewcommand{\headrulewidth}{0.5pt}  % change top line
\renewcommand{\footrulewidth}{0.5pt}  % show bottom line

% no indent, no extra space between paragraphs, line spacing
\setlength{\parindent}{0cm}
\setlength{\parskip}{0cm}
\renewcommand{\baselinestretch}{1.15}

% deal with hbox overflow
\usepackage[activate={true, nocompatibility}, final, tracking=true, kerning=true, spacing=true,
            factor=1100, stretch=10, shrink=10]{microtype}

% bibliography
\usepackage[authordate, backend=biber, sorting=nty]{biblatex-chicago}
\addbibresource{sources.bib}

% load external figures, embed functioning links
\usepackage{graphicx, hyperref}

% sub-figures
\usepackage{caption, subcaption}

% math symbols
\usepackage{amsmath, amssymb}

%%%%%% stuff before main part
\begin{document}
\fancyhf{}  % clear standards
\begin{titlepage}
\centering
\Huge

Seminar Paper \\
Topic

\vspace{2cm}
\includegraphics[width=0.666\textwidth]{Figures/Sigillum.png} \\
\vspace{2cm}

\begin{tabular}{ll}
Author & Maximilian Schneider \\
Supervisor & Max Mustermann \\
\end{tabular}
\vspace{1cm}

Department Of Statistics \\
Ludwig-Maximilians-Universität München \\
\today

\end{titlepage}


\newpage
\begin{abstract}
\setlength{\parindent}{0cm}
\setlength{\parskip}{0cm}
\renewcommand{\baselinestretch}{1.15}
\noindent
Lorem ipsum dolor sit amet, consetetur sadipscing elitr, sed diam nonumy eirmod tempor invidunt ut labore et dolore magna aliquyam erat, sed diam voluptua.
At vero eos et accusam et justo duo dolores et ea rebum.
Stet clita kasd gubergren, no sea takimata sanctus est Lorem ipsum dolor sit amet.
Lorem ipsum dolor sit amet, consetetur sadipscing elitr, sed diam nonumy eirmod tempor invidunt ut labore et dolore magna aliquyam erat, sed diam voluptua.
At vero eos et accusam et justo duo dolores et ea rebum. Stet clita kasd gubergren, no sea takimata sanctus est Lorem ipsum dolor sit amet.

Lorem ipsum dolor sit amet, consetetur sadipscing elitr, sed diam nonumy eirmod tempor invidunt ut labore et dolore magna aliquyam erat, seLorem ipsum dolor sit amet, consetetur sadipscing elitr, sed diam nonumy eirmod tempor invidunt ut labore et dolore magna aliquyam erat, sed diam voluptua.
At vero eos et accusam et justo duo dolores et ea rebum.
Stet clita kasd gubergren, no sea takimata sanctus est Lorem ipsum dolor sit amet.
Lorem ipsum dolor sit amet, consetetur sadipscing elitr, sed diam nonumy eirmod tempor invidunt ut labore et dolore magna aliquyam erat, sed diam voluptua.
At vero eos et accusam et justo duo dolores et ea rebum. Stet clita kasd gubergren, no sea takimata sanctus est Lorem ipsum dolor sit amet.
\end{abstract}


\newpage
\fancyhead[r]{\leftmark}  % chapter on top right
% \renewcommand{\contentsname}{Table of Contents}  % if i want to rename it
\tableofcontents

%%%%% main part
\newpage

% layout for main part
\fancyfoot[C]{\thepage}  % page number
\setcounter{page}{1}

\section{Section One}
\textcite{src1} and \textcite{scr2}. \\
Lorem ipsum dolor sit amet, consetetur sadipscing elitr, sed diam nonumy eirmod tempor invidunt ut labore et dolore magna aliquyam erat, sed diam voluptua.
At vero eos et accusam et justo duo dolores et ea rebum.

\subsection{Subsection One One}
Lorem ipsum dolor sit amet, consetetur sadipscing elitr, sed diam nonumy eirmod tempor invidunt ut labore et dolore magna aliquyam erat, sed diam voluptua.
At vero eos et accusam et justo duo dolores et ea rebum.


\section{Section Two}
Lorem ipsum dolor sit amet, consetetur sadipscing elitr, sed diam nonumy eirmod tempor invidunt ut labore et dolore magna aliquyam erat, sed diam voluptua.
At vero eos et accusam et justo duo dolores et ea rebum.

\subsection{Subsection Two One}
Lorem ipsum dolor sit amet, consetetur sadipscing elitr, sed diam nonumy eirmod tempor invidunt ut labore et dolore magna aliquyam erat, sed diam voluptua.
At vero eos et accusam et justo duo dolores et ea rebum.


\input{Sections/03-section}

%%%%% stuff after main part
\appendix
\newpage
\fancyfoot[C]{}  % remove page number
\printbibliography[heading=bibnumbered, title = Bibliography]

\begin{comment}
\newpage
\section{List Of Figures}
\listoffigures
\end{comment}

\newpage
\input{Appendix/appendix}

\end{document}
